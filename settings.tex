% !Mode:: "TeX:UTF-8"
\usepackage{comment}            %块注释
\usepackage{indentfirst}         %首行缩进
\usepackage{type1cm}          %控制字体大小
\usepackage{color, xcolor}    %支持彩色文本、底色、文本框等
\usepackage{bbding}                  % 一些特殊符号

\usepackage{makeidx}                 % 建立索引

\usepackage{listings}                % 粘贴源代码
\lstloadlanguages{}                  % 所要粘贴代码的编程语言
\lstset{language=,tabsize=4, keepspaces=true,
    xleftmargin=2em,xrightmargin=2em, aboveskip=1em,
    backgroundcolor=\color{lightgray},    % 定义背景颜色
    frame=none,                      % 表示不要边框
    keywordstyle=\color{blue}\bfseries,
    breakindent=22pt,
    numbers=left,stepnumber=1,numberstyle=\tiny,
    basicstyle=\footnotesize,
    showspaces=false,
    flexiblecolumns=true,
    breaklines=true, breakautoindent=true,breakindent=4em,
    escapeinside={/*@}{@*/}
}

%%%%%%%%%% 数学符号公式 %%%%%%%%%%
\usepackage{latexsym}
\usepackage{amsmath}                 % AMS LaTeX宏包
\usepackage{amssymb}                 % 用来排版漂亮的数学公式
\usepackage{amsbsy}
\usepackage{amsthm}
\usepackage{amsfonts}
\usepackage{mathrsfs}                % 英文花体字体
\usepackage{bm}                      % 数学公式中的黑斜体
\usepackage{relsize}                 % 调整公式字体大小:\mathsmaller, \mathlarger
\usepackage{caption2}                % 浮动图形和表格标题样式

\ctexset{
	section = {
		name = {第,节},
		format = \Large\bfseries\raggedright
	},
}
\makeindex    % 生成索引
