% !Mode:: "TeX:UTF-8" 

\begin{abstract}
\noindent  %不缩进
这是一个简单的Xe\LaTeX{}+CJK的模板,为\TeX{}的初学者提供便利上手的参照。该模板在Win7+Xe\LaTeX{}下编译通过,适合在Windows下工作的朋友。从一个简单的模板出发,不断地提升对\TeX{}的认识,培养良好的写作风格。网上有大量的资料,我推荐\LaTeX{}编辑部,那里能找到国内外许多期刊的模板和一些高校博/硕士论文的模板。祝玩儿得开心!
\end{abstract}

\PencilRightUp % 一些可爱的图标,需要bbding宏包的支持
公元1974年,ACM图灵奖授予了Standford大学教授\index{Donald E. Knuth}Donald E. Knuth(高德纳),表彰他在算法和程序语言设计等多方面杰出的成就。他的巨著The Art of Computer Programming令人震撼,感兴趣的读者可以访问他的主页http://www-cs-faculty.stanford.edu/$\sim$knuth/index.html。
另外,Knuth~的突出贡献还包括\index{\TeX{}系统}\TeX{}系统,毫不夸张地评价,\TeX{}给科技论文的排版带来了一场革命。

%%%%%%%%%% section %%%%%%%%%%
\section{编辑数学公式}

\indent   % 恢复缩进
\TeX{}有诸如AMS\TeX、\LaTeX{}等宏库。在FreeBSD下,缺省的宏库是te\TeX。Knuth用\$符号界定数学公式,意味着每个好的公式都是无价之宝。有了\TeX{}系统,输入数学公式变得简单愉快。如,
\begin{theorem}[L\'{e}vy\index{L\'{e}vy定理}]
令$F(x),\varphi(t)$分别为随机变量$X$的分布函数和特征函数。
假定$F(x)$在$a+h$和$a-h (h>0)$处连续,则有
\begin{eqnarray}
  \label{Levy theorem}  % 方程的标记可以是专有名词
F(a+h)-F(a-h)&=&\lim_{T\rightarrow\infty} \frac{1}{\pi}\int^{T}_{-T} \frac{\sin ht}{t} e^{-ita} \varphi(t)dt
\end{eqnarray}
\end{theorem}
\begin{proof}
  从略。感兴趣的读者可以参考……。
\end{proof}
L\'{e}vy定理在分布函数和特征函数之间搭建了一座桥梁。由公式(\ref{Levy theorem})可得
\begin{eqnarray}
  \label{DensityCharacteristic}   % 自定义的标记
  f(x)&=&\frac{1}{2\pi}\int^{+\infty}_{-\infty} e^{-itx}\varphi(t)dt
\end{eqnarray}
\begin{proof}
由(\ref{Levy theorem})和Lebesgue定理,我们有
\begin{eqnarray}
  \frac{F(x+\Delta x)-F(x)}{\Delta x}&=&\frac{1}{2\pi} \int^{+\infty}_{-\infty}
\frac{\sin(t\Delta x/2)}{t\Delta x/2} e^{-it(x+\Delta x/2)} \varphi(t) dt\nonumber\\
  f(x)&=&\frac{1}{2\pi} \int^{+\infty}_{-\infty} \lim_{\Delta x\rightarrow 0}
\frac{\sin(t\Delta x/2)}{t\Delta x/2} e^{-it(x+\Delta x/2)} \varphi(t) dt\nonumber\\
  &=&\frac{1}{2\pi}\int^{+\infty}_{-\infty} e^{-itx}\varphi(t)dt\nonumber
\end{eqnarray}
我们知道特征函数的定义是
\begin{eqnarray}
  \label{section1:characteristic}   % 标记中注明了章节号
  \varphi(t)&=& E(e^{itX})\nonumber\\
            &=& \int^{+\infty}_{-\infty} e^{itx} f(x)dx
\end{eqnarray}
对比(\ref{DensityCharacteristic})和(\ref{section1:characteristic})可见,密度函数和特征函数之间的关系非常巧妙。
\end{proof}
\HandRight 在\TeX{}环境里,数学公式的表达是很自然的,绝大多数命令就是英文的数学
专有名词或它们的缩写,如果你以前读过英文的数学文献,记忆这些命令是不难的。手头有个命令快速寻查表是很方便的,
我用的是~Hypertext Help with \LaTeX,网上可以搜到,是免费的。



%%%%%%%%%% section %%%%%%%%%%
\section{符号、字体、颜色等}
\begin{itemize}
\item 特殊字符:\# \$ \% \^{} \& \_ \{ \} \~{} $\backslash \cdots$

\item 特殊字符:\# \$ \% \^{A} \& \_ \{ \} \~{A} ? \^{?} \~{?} $\backslash \cdots$

\item 中文字体:{\songti 宋体} {\kaishu 楷书} {\fangsong 仿宋} {\heiti 黑体} {\youyuan 幼圆} {\lishu 隶书} {\yahei 雅黑}

\item 文字大小:{\tiny tiny} {\scriptsize script size} {\footnotesize footnote size} {\small small} {\normalsize normal size} {\large large} {\Large larger} {\LARGE even larger} {\huge huge} {\Huge largest}

\item 各种颜色:{\color{red} 红色} {\color{yellow} 黄色} {\color{blue} 蓝色} {\color{green} 绿色} {\color{magenta}  洋红} {\color{cyan} 蓝绿}
\end{itemize}



%%%%%%%%%% section %%%%%%%%%%
\section{图形表格等浮动对象}

\index{贝叶斯方法}贝叶斯方法\cite{Gelman}主要用于小样本数据分析,它利用参数先验分布和
后验分布之差异进行统计推断,其一般步骤是:
\begin{enumerate}
  \item 构建概率模型,包括参数的先验分布。
  \item 给定观察数据,计算参数的后验分布。
  \item 分析模型的效果,如有必要,回到第一步。
\end{enumerate}
下面,我们给一个表格的例子:
\begin{center}
\begin{table}[!h]     % 强制在原位显示表格
\centering
\caption{二维随机向量$(X,Y)$的边缘分布}
\begin{tabular}{l|ccccc|c}
  $_X$\hspace{3mm} $^Y$&$y_1$&$y_2$&$\cdots$&$y_j$&$\cdots$\\
\hline
$x_1$   &$p_{11}$&$p_{12}$&$\cdots$&$p_{1j}$&$\cdots$&$p_{1\cdot}$\\
$x_2$   &$p_{21}$&$p_{22}$&$\cdots$&$p_{2j}$&$\cdots$&$p_{2\cdot}$\\
$\vdots$&$\vdots$&$\vdots$&$\vdots$&$\vdots$&$\vdots$&$\vdots$ \\
$x_i$   &$p_{i1}$&$p_{i2}$&$\cdots$&$p_{ij}$&$\cdots$&$p_{i\cdot}$\\
$\vdots$&$\vdots$&$\vdots$&$\vdots$&$\vdots$&$\vdots$&$\vdots$ \\
\hline
   &$p_{\cdot 1}$&$p_{\cdot 2}$&$\cdots$&$p_{\cdot j}$&$\cdots$&1
\label{marginal distribution}
\end{tabular}
\end{table}
\end{center}
在表\ref{marginal distribution}中,$p_{\cdot j}=\sum\limits_i p_{ij}$,类似地,$ p_{i\cdot}=\sum\limits_j p_{ij}$。
% 插入一个图片
%\includegraphics[width=50mm,height=40mm]{figures/demo.eps}

%%%%%%%%%% section %%%%%%%%%%
\section{先把问题记录在这里吧}
\begin{itemize}
\item 为什么没有加那个tableofcontents语句就自动生成目录了呢?好像又不是目录,只有第1节的页数索引
\item xe\LaTeX{}到底是个什么东西?
\item 中文和英文之间的间距不和谐。据说可以用波浪号来解决?
\item 关于中文字体,就先列举上面几个。详细的内容可以进一步研究。
\item 为什么在没有把Contents重定义成目录(renewcommand contentsname 目录)的情况下,用tableofcontents命令后生成的 Contents 自动变成了中文的 目录 ?
\item 有点晚了,关于字号下次再说吧。
\end{itemize}

%%%%%%%%%% section %%%%%%%%%%
\section{生成索引}
三次编译:xelatex+makeindex+xelatex  文件名。\\
\indent 譬如对这个模板,生成Template4DOC.ind的过程如下。
\begin{lstlisting}
$ makeindex Template4CJK
This is makeindex, version 2.14 [02-Oct-2002] (kpathsea + Thai support).
Scanning input file Template4CJK.idx....done (4 entries accepted, 0 rejected).
Sorting entries....done (9 comparisons).
Generating output file Template4CJK.ind....done (18 lines written, 0 warnings).
Output written in Template4CJK.ind.
Transcript written in Template4CJK.ilg.
\end{lstlisting}

\printindex % 打印出索引名及其所在页码,即那些\index{索引名}
%%%%%%%%%% 参考文献 %%%%%%%%%%
\begin{thebibliography}{}
\bibitem[Gelman et~al., 2004]{Gelman} Gelman, A., Carlin, J.~B., Stern, H.~S.  \& Rubin, D.~B. (2004)
Bayesian Data Analysis (Second Edition).  \newblock Chapman \& Hall/CRC.
\end{thebibliography}

\clearpage